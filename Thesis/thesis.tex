% This is the Reed College LaTeX thesis template. Most of the work 
% for the document class was done by Sam Noble (SN), as well as this
% template. Later comments etc. by Ben Salzberg (BTS). Additional
% restructuring and APA support by Jess Youngberg (JY).
% Your comments and suggestions are more than welcome; please email
% them to cus@reed.edu
%
% See http://web.reed.edu/cis/help/latex.html for help. There are a 
% great bunch of help pages there, with notes on
% getting started, bibtex, etc. Go there and read it if you're not
% already familiar with LaTeX.
%
% Any line that starts with a percent symbol is a comment. 
% They won't show up in the document, and are useful for notes 
% to yourself and explaining commands. 
% Commenting also removes a line from the document; 
% very handy for troubleshooting problems. -BTS

% As far as I know, this follows the requirements laid out in 
% the 2002-2003 Senior Handbook. Ask a librarian to check the 
% document before binding. -SN

%%
%% Preamble
%%
% \documentclass{<something>} must begin each LaTeX document
\documentclass[12pt,twoside]{reedthesis}
% Packages are extensions to the basic LaTeX functions. Whatever you
% want to typeset, there is probably a package out there for it.
% Chemistry (chemtex), screenplays, you name it.
% Check out CTAN to see: http://www.ctan.org/
%%
\usepackage{graphicx,latexsym} 
\usepackage{amssymb,amsthm,amsmath}
\usepackage{longtable,booktabs,setspace} 
\usepackage{chemarr} %% Useful for one reaction arrow, useless if you're not a chem major
\usepackage[hyphens]{url}
\usepackage{rotating}
\usepackage{natbib}
% Comment out the natbib line above and uncomment the following two lines to use the new 
% biblatex-chicago style, for Chicago A. Also make some changes at the end where the 
% bibliography is included. 
%\usepackage{biblatex-chicago}
%\bibliography{thesis}

% \usepackage{times} % other fonts are available like times, bookman, charter, palatino

\title{My Final College Paper}
\author{Dylan H. Huff}
% The month and year that you submit your FINAL draft TO THE LIBRARY (May or December)
\date{May 2019}
\division{Mathematics and Natural Sciences}
\advisor{Eric S. Roberts}
%If you have two advisors for some reason, you can use the following
%\altadvisor{Your Other Advisor}
%%% Remember to use the correct department!
\department{Computer Science}
% if you're writing a thesis in an interdisciplinary major,
% uncomment the line below and change the text as appropriate.
% check the Senior Handbook if unsure.
%\thedivisionof{The Established Interdisciplinary Committee for}
% if you want the approval page to say "Approved for the Committee",
% uncomment the next line
%\approvedforthe{Committee}

\setlength{\parskip}{0pt}
%% End Preamble
%%
%% The fun begins:
\begin{document}

  \maketitle
  \frontmatter % this stuff will be roman-numbered
  \pagestyle{empty} % this removes page numbers from the frontmatter

% Acknowledgements (Acceptable American spelling) are optional
% So are Acknowledgments (proper English spelling)
    \chapter*{Acknowledgements}
	I want to thank a few people.

% The preface is optional
% To remove it, comment it out or delete it.
    \chapter*{Preface}
	This is an example of a thesis setup to use the reed thesis document class.
	
	

    \chapter*{List of Abbreviations}
		You can always change the way your abbreviations are formatted. Play around with it yourself, use tables, or come to CUS if you'd like to change the way it looks. You can also completely remove this chapter if you have no need for a list of abbreviations. Here is an example of what this could look like:

	\begin{table}[h]
	\centering % You could remove this to move table to the left
	\begin{tabular}{ll}
		\textbf{CS}   & Computer Science\\
		\textbf{JS}  	&  JavaScript\\
	\end{tabular}
	\end{table}
	

    \tableofcontents
% if you want a list of tables, optional
    \listoftables
% if you want a list of figures, also optional
    \listoffigures

% The abstract is not required if you're writing a creative thesis (but aren't they all?)
% If your abstract is longer than a page, there may be a formatting issue.
    \chapter*{Abstract}
	The preface pretty much says it all.
	
	\chapter*{Dedication}
	You can have a dedication here if you wish.

  \mainmatter % here the regular arabic numbering starts
  \pagestyle{fancyplain} % turns page numbering back on

%The \introduction command is provided as a convenience.
%if you want special chapter formatting, you'll probably want to avoid using it altogether

    \chapter*{Introduction}
         \addcontentsline{toc}{chapter}{Introduction}
	\chaptermark{Introduction}
	\markboth{Introduction}{Introduction}

\chapter{Algorithm Animation}
This chapter is an overview of algorithm animation and where this thesis fits into that story. 
	
\section{What is Algorithm Animation?}
Algorithm animation is the process of taking algorithms and giving them graphical representations.
 
\section{Motivation for Algorithm Animation}
Algorithm animation has benefits for teachers and students. Below are some of the most commonly cited benefits.\footnote{\cite{hundhausen_meta-study_2002}}
\begin{itemize}
\item It allows teachers to display algorithms in lectures easily.
\item Another method for teaching students fundamental algorithms, pictures and code compared to just code.
\item It allows for another avenue of debugging. 
\end{itemize}


\section{Use in Computer Science Education}

\subsection{History}
Algorithm animation began in the 70's and has seen usage since. In the early days of algorithm animation, teachers used tools to make animations for their presentations. Often time these were predefined short films. As tools progressed, the animations no longer had to be used exclusively by teachers and didn't need to be predefined.\footnote{\cite{hundhausen_meta-study_2002}} Tools became able to dynamically represent the algorithms that students made. 

One of the most famous and important contributions to algorithm animation was BALSA. BALSA was created in 1987 by Marc Brown. BALSA introduced several major innovations in the field. One major contribution was the addition of real time animations. Prior to BALSA, the animations wouldn't operate in real time, rather the animation would algorithm would run then the animation would be made, unlike real time animation which executes the animation and algorithm simultaneously. Another innovation was the introduction of scripts. Scripts were predefined PASCAL programs that would control the algorithm and could be executed in real time. This allows teachers to predefine how an animation will execute, then present the animation in real time.\footnote{\cite{brown_algorithm_1987}}

Along with this shift form predefined to dynamically defined tools, other features were added that improved the animations. One paritcularly notable improvment came in when animations transitioned from 2D to 3D. This shift from 2D to 3D allowed for more information to be simultaneously displayed. Along with this improvement, color and sound were both added to animations.\footnote{\cite{najork_library_1994}}

\subsection{Current State of Algorithm Animation}
There are a few challenges facing modern day algorithm animation. One of these challenges is the lack of adoption of animation tools by instructors.Levy points out two main challenges of adoption from survey results of teachers. The first is that the tools being developed may be feature rich but not integrated well into the existing material or curriculum. This illuminates the fact that the tool developers aren't usually primarily concerned with integration, but rather features. Second, they cite "centrality" as the other major inhibition of teachers. They define centrality to be 	where the center of learning is for the students. By making animation tools that animate the students algorithms, the centrality is being moved from the teacher to the student. They note that this phenomenon is present with highly confident and experienced teachers through not confidant inexperienced teachers.\footnote{\cite{levy_we_2007}}  Its also worth noting that the teachers in this study are high school teachers and not college professors. 

Along with low adoption rates by teachers, there is also a lack of work being published.\footnote{\cite{kucera_visualization_2018}}  


\chapter{State of Computer Science Higher Education}	
This chapter will be about the state of CS higher ed, its problems, some solutions and how JavaPPTX fits into that picture

\section{Background}
Talk about the history and formation. Talk about its periods of growth and bust. Talk about current growth and future growth. 

\section{Problems facing Higher Education}
Talk about various problems, how they arose and what their effects on education are.

\subsection{Lack of PhDs}
lack of number of people and huge demand for PhDs from market

\subsection{Increasing Demand from Students}


\section{Proposed Solutions}
In each subsection, I will talk about the pros, cons and feasibility of every solution presented

\subsection{Retraining PhDs}
taking teachers with PhDs not in CS and retraining them to teach intro level classes.

\subsection{Teaching Only Staff}
Staff that teach but don't hold the professor title

\subsection{Teaching Tools}
talk about some of the existing tools to aid profs.


\section{JavaPPTX}
Overview of the package, how it helps address some of the problems in CS higher ed, and I'll end with talking about areas of expansion, which will be a good segway into my project. Also mention that this tool, and no tool, will be the solution to solving the problems facing higher ed.

\subsection{Background}
Here I will briefly talk about the features in the package. Some features include saving time animating and more effective at conveying concepts to students using pptx.


\subsection{Classroom Usage}
I'll talk about how it fits in unobtrusively and its current usage. May not be enough here to warrant a subsection. 
	
\subsection{Package Expansion}
Here I'll talk about ways that the package could be expanded and the benefits of the specific expansions. One will be my expansion. 

\chapter{JavaPPTX to JavaScript}

\section{Statement of Work}

\subsection{New Features}
Talk about the addition of native output to hmtl, css and js. Talk about the ease of turning this into a faculty webpage without any new things being installed

\subsection{Uses of this Expansion}
How its useful to have a native JS output compared to PPTX. Also the pptx and js will be identical

\section{Internal Logic}
This section will explain, at a high level, the logic of what I did. 

\subsection{Cross Compilation}
Talk about working through the existing logic and translating that. 

\subsection{JavaScript Framework}
Talk about the construction of the JS framework and its features.

\section{Future Work}

\subsection{New Features}
What features did I miss? How can this overall be improved? 

\subsection{Realtime JS Animation}
Use the JS framework that I built to make a real time animation package. 


	
\chapter*{Conclusion}
         \addcontentsline{toc}{chapter}{Conclusion}
	\chaptermark{Conclusion}
	\markboth{Conclusion}{Conclusion}
	\setcounter{chapter}{4}
	\setcounter{section}{0}
	
Here's a conclusion
    \appendix
      \chapter{The First Appendix}


%This is where endnotes are supposed to go, if you have them.
%I have no idea how endnotes work with LaTeX.

  \backmatter % backmatter makes the index and bibliography appear properly in the t.o.c...

% if you're using bibtex, the next line forces every entry in the bibtex file to be included
% in your bibliography, regardless of whether or not you've cited it in the thesis.
    \nocite{*}

% Rename my bibliography to be called "Works Cited" and not "References" or ``Bibliography''
% \renewcommand{\bibname}{Works Cited}

%    \bibliographystyle{bsts/mla-good} % there are a variety of styles available; 
%  \bibliographystyle{plainnat}
% replace ``plainnat'' with the style of choice. You can refer to files in the bsts or APA 
% subfolder, e.g. 
 \bibliographystyle{APA/apa-good}  % or
 \bibliography{thesis}
 % Comment the above two lines and uncomment the next line to use biblatex-chicago.
 %\printbibliography[heading=bibintoc]

% Finally, an index would go here... but it is also optional.
\end{document}
